\documentclass[12pt,letterpaper,titlepage,en-US]{article}

\usepackage{basicstyle}
\usepackage{report}
%\usepackage{knit}

\usepackage{listings}
\usepackage{xcolor}


\definecolor{codegreen}{rgb}{0,0.6,0}
\definecolor{codegray}{rgb}{0.5,0.5,0.5}
\definecolor{codepurple}{rgb}{0.58,0,0.82}
\definecolor{backcolour}{rgb}{0.95,0.95,0.92}
 
\lstdefinestyle{mystyle}{
    backgroundcolor=\color{backcolour},   
    commentstyle=\color{codegreen},
    keywordstyle=\color{magenta},
    numberstyle=\tiny\color{codegray},
    stringstyle=\color{codepurple},
    basicstyle=\ttfamily\footnotesize,
    breakatwhitespace=false,         
    breaklines=true,                 
    captionpos=b,                    
    keepspaces=true,                 
    numbers=left,                    
    numbersep=5pt,                  
    showspaces=false,                
    showstringspaces=false,
    showtabs=false,                  
    tabsize=2
}
 
\lstset{style=mystyle}



\usepackage[toc,page]{appendix}

\newcommand{\hmwkTitle}{Project \#3}
\DTMsavetimestamp{DueDate}{2019-11-27T11:59:00+00:00}
\newcommand{\hmwkClass}{CS 6385.001}
\newcommand{\hmwkClassName}{Algorithmic Aspects of Telecommunication Networks}
\newcommand{\hmwkClassInstructor}{Instructor: Prof. Andras Farago}
\newcommand{\hmwkAuthorName}{Shyam Patharla}
\newcommand{\hmwkAuthorNetID}{sxp178231}




%
% Title Page
%

\title{
    \vspace{1in}
    \textmd{\textbf{\hmwkClassName \\\hmwkClass:\ \hmwkTitle }}\\
    \normalsize\vspace{0.1in}\small{Due\ on\ \DTMusedate{DueDate}\ at \DTMusetime{DueDate} }\\
    \vspace{0.1in}\large{\textit{\hmwkClassInstructor}}\\
    \vspace{0.5in}\includegraphics[height=2.4em]{UTD_logo_BW}\\
    \vspace{2in}
}

\author{\textbf{\hmwkAuthorName\ \footnotesize{(\hmwkAuthorNetID)}} \\ }
\date{}
\makeindex

\begin{document}
\maketitle
\pagenumbering{Roman}

\tableofcontents

\pagebreak
\pagenumbering{arabic}

\section{Introduction}
\begin{itemize}
\item In this project, we try to implement the Nagamochi Ibaraki algorithm for finding the edge connectivity of a connected graph
\item Give an input graph with n nodes and m edges, our program outputs the edge connectivity of the graph
\item We do this for range of values of m
\item  We also analyze how the edge connectivity varies with the value of m and assess the spread of the edge connectivity values
\item We discuss the possible reasons for the above characteristics.
\end{itemize}

\section{Design Decisions}
\begin{itemize}
\item We implement the solution in the \textbf{Java} programming language
\item The program modules were run on a \textbf{Mac} operating system

\end{itemize}





\section{Solution Approach}

\subsection{Generating Input Examples}
We first generate  input graphs for simulating the algorithm.
These graphs are then passed on to second \textit{Module 1} module which runs the Exhaustive Enumeration algorithm on them. The graphs are generated as follows.

\begin{itemize}
\item For all examples, we set the number of nodes in the network to 5 i.e \textit{n} = 5

\item The topology of the graph for all examples is a complete graph, hence the numbe of edges m=10



\end{itemize}




\subsection{Exhaustive Enumeration Algorithm}
\textit{Module 2} receives input parameters from the first module (graph). It has the following functions.


\begin{itemize}
\item The n nodes in our network are always up
\item We have m links each of which may fail
\item The minimum number of links that may fail is zero, while the maximum number is m
\item We can generate $2^{m}$ possible states, each of which may contain some links which fail and some links do not
\item We proceed step by step, generating states with k failures, k=0,1,2,...m and store them
\item We generate 
\end{itemize}

Module 2  passes the generated states it on to Module 3, which computes the raliability for the network.



\subsection{Reliability}

\textit{Module 3} takes the output parameters of \textit{Module 2} i.e. the \textbf{generated states} for an input graph.

\begin{itemize}


\item We iterate through the states received

 \item For each state we check if the network is \textit{connected} using a depth first search
 
 \item  If it is connected, then the system condition is \textbf{up}, else the system condition is \textit{down}

\item We compute the \textbf{probability} of the state \textit{s} using the formula
\begin{equation}
prob(s)=p*u + (1-p)*d
\end{equation}

where u is the number of links which are up and d is the number of links which are down

\item The \textbf{reliability} of the network is the sum of probabilities of the \textbf{up} states
 
\end{itemize}


\subsection{Presentation of Results}
\begin{itemize}

\item We use a value \textbf{p} to denote the reliability of a component (link) 

\item We also use a parameter \textbf{k} which denotes the number of states (randomly chosen) whose condition is flipped i.e. from up to down or vice versa

\item \textit{Task 1} consists of varying the value of \textbf{p} and seeing how the reliability of the network changes

\item We vary the value of p  from 0 to 1 in steps of 0.04  

 \includegraphics[scale=0.20]{fig/task1.png}
 
 

\item \textit{Task 2} consists of fixing the value of \textbf{p}, varying \textbf{k} and seeing how the reliability of the network varies

\item We fix the value of p to 0.87 and vary the value of k from 0 to 25 in steps of 1 

\item We take 5 trials for each value of k and take the average of the reliabilities
 \end{itemize}

The flowchart for Task 2 is shown below.

 \includegraphics[scale=0.20]{fig/task2.png}


\section{Exhaustive Enumeration Algorithm - Explanation}
\begin{itemize}


\item The goal is to find the \textit{reliability} of a network where nodes anre always up but links may fail

\item We generate \textbf{all possible states} for the network where each component may be \textit{up} or \textit{down}

\item For each state, we check if the network is \textbf{connected}

\item If it is connected we compute its probability using the formula
\begin{equation}
prob(s)=p*u + (1-p)*d
\end{equation}

where \textbf{u} is the number of links which are up and \textbf{d} is the number of links which are down


\item The \textbf{reliability} of the network is the sum of probabilities of all \textit{up} states





\end{itemize}








\begin{algorithm}[H]
    \caption{ExhaustiveEnumeration}
    \begin{algorithmic}[1]
        \Procedure{ExhaustiveEnumeration}{$V,E,p$}
      
      \State $m= \mid E \mid$
      \State $states[][] = generateStates(m)$
      
      \State $t = 2^m $
     \State $reliability=0$
      
      
      \For {$i = 1 \text{ to } t$}
      	
      	\State $ s = states[i]$
      	\State $fail=0$
      	
      	\For {$j=1 \text{ to } m$}
      		\If {$s[i]=1$}
      			\State $fail=fail+1$
      			\EndIf
      			\EndFor
      			
      	\State $prob=p*(m-fail)+(1-p)*fail$
      	\State $G' = transform(V,E,s)$
      	\If {$isConnected(G')$}
      	\State $reliability=reliability+prob$
      	\EndIf
      	
      	
      \EndFor
      
        
      
     
     
      return reliability
        \EndProcedure
    \end{algorithmic}
    \end{algorithm}




\section{Observations and Analysis}
\begin{itemize}


\item The program produces outputs as shown in the figures below.\\


 
\item The output results are stored in a \textit{csv} file. The graphs are generated in \textbf{R}.

\item We plot the graph of the \textit{number of edges} m vs. the \textit{edge connectivity} $\lambda(G)$

\includegraphics[scale=0.6]{fig/plot1.png}\\

\item We can clearly see that the edge connectivity of the network \textbf{increases} with increase in the value of m



\item We plot the graph of edge connectivity values \textbf{lambda(G)} against their \textbf{spread}\\

\includegraphics[scale=0.6]{fig/plot2.png}




\item We can clearly see that the \textbf{spread} of edge connectivity values increases, saturates in the middle and then decreases
\end{itemize}
 
\section{Discussion}
\begin{itemize}

\item As the number of edges in the graph increases, the number of edge disjoint paths between any two nodes tends to increase, hence the edge connectivity of the graph also \textbf{increases}

\item In our observations, the highest values of spread occur for $\lambda(G)=2,3,4$ i.e middle values with decreasing values on either side

\item The reason is that a given value of  edges connectivity (especially the ones in the middle) tends to persist for a range of m values, especially since for each graph we choose edges randomly, which sort of randomises our progression as m increases



\end{itemize}





  \section{ReadMe File}
  This section shows how to run the project files.
  \begin{itemize}
  \item Downloads the project files and store them in a folder
  \item Open the project folder in Eclipse
  \item Open the file \textbf{Presentation.java}
  \item Right Click $->$ Run as $->$ Java Application
  \item Alternatively,navigate to the folder in \textbf{terminal} and run the following commands
  \begin{itemize}
  \item javac Presentation.java
  \item java Presentation
  \end{itemize}
  \end{itemize}

  

 
\section{Code}

\textbf{Module 1: ExhaustiveEnumeration.java}
\lstinputlisting{/users/psprao/eclipse-workspace/network-reliability/src/ExhaustiveEnumeration.java}

\pagebreak
\textbf{Module 2: Reliability.java}
\lstinputlisting{/users/psprao/eclipse-workspace/network-reliability/src/Reliability.java}

\pagebreak
\textbf{Module 3: Presentation.java}
\lstinputlisting{/users/psprao/eclipse-workspace/network-reliability/src/Analysis.java}

\pagebreak
\textbf{Utils.java}
\lstinputlisting{/users/psprao/eclipse-workspace/network-reliability/src/Utils.java}



\textbf{Visualization.R.java}
\lstinputlisting{/users/psprao/eclipse-workspace/Nagamochi-Ibaraki/visualization.R}





\end{document}
